\documentclass[12pt,a4paper]{article}
\usepackage[utf8]{inputenc}
\usepackage[T2A]{fontenc}
\usepackage[russian]{babel}
\usepackage{graphicx}
\usepackage{geometry}
\usepackage{float}
\usepackage{hyperref}
\geometry{left=2cm, right=2cm, top=2cm, bottom=2cm}

\title{Отчет по безопасности и архитектуре}
\author{Горячев Иван Сергеевич \\ Группа: БПИ236}
\date{Дата: 02.11.2025 }

\begin{document}

\maketitle

\section{NFR - 8 требований с критериями}

В системе реализовано 8 нефункциональных требований безопасности. Каждое требование имеет четкие критерии и метрики.

\begin{figure}[H]
\centering
\includegraphics[width=0.9\textwidth]{nfr_table.png}
\caption{Таблица нефункциональных требований}
\end{figure}

\textbf{Реализация:}
\begin{itemize}
\item NFR-01: JWT токены с TTL 24 часа
\item NFR-02: RBAC система авторизации
\item NFR-03: Хеширование паролей с Argon2id
\item NFR-04: SQLAlchemy ORM для защиты от инъекций
\item NFR-05: Логирование всех попыток доступа
\item NFR-06: Хранение логов 30 дней
\item NFR-07: Мониторинг и автоматическое восстановление
\item NFR-08: Мониторинг времени ответа API
\end{itemize}

\section{DFD - 3-4 компонента с границами доверия}

Диаграмма потоков данных показывает архитектуру системы с четкими границами доверия.

\begin{figure}[H]
\centering
\includegraphics[width=0.8\textwidth]{flow_chart.png}
\caption{Диаграмма потоков данных}
\end{figure}

\textbf{Компоненты и границы:}
\begin{itemize}
\item \textbf{Ненадежная зона:} Пользователь, Браузер
\item \textbf{Edge зона:} Load Balancer (доверенная)
\item \textbf{Backend зона:} API, Аутентификация, Бизнес-логика (доверенная)
\item \textbf{Data зона:} БД, Файлы, Логи (доверенная)
\end{itemize}

\section{STRIDE - 16 угроз безопасности}

Проведен анализ угроз по методике STRIDE. Выявлено 16 угроз.

\begin{figure}[H]
\centering
\includegraphics[width=0.9\textwidth]{stride_table.png}
\caption{Таблица анализа угроз STRIDE}
\end{figure}

\textbf{Ключевые угрозы:}
\begin{itemize}
\item Подделка HTTP запросов (S)
\item SQL инъекции (T)
\item Отказ в обслуживании (D)
\item Несанкционированный доступ (I)
\end{itemize}

\section{Трассируемость мер безопасности}

Все меры безопасности имеют четкую трассировку к требованиям и рискам.

\begin{figure}[H]
\centering
\includegraphics[width=0.9\textwidth]{traceability_table.png}
\caption{Трассируемость требований к задачам}
\end{figure}

\textbf{Архитектурные меры:}
\begin{itemize}
\item Слоистая архитектура → Изоляция бизнес-логики
\item Инъекция зависимостей → Контроль доступа
\item Конфигурация через env → Защита секретов
\item SQLAlchemy ORM → Защита от SQL-инъекций
\end{itemize}

\section{Приоритизация рисков}

Риски приоритизированы по вероятности и влиянию.

\begin{figure}[H]
\centering
\includegraphics[width=0.9\textwidth]{risks_table.png}
\caption{Таблица приоритизации рисков}
\end{figure}

\textbf{Критические риски:}
\begin{itemize}
\item R1: Утечка учетных данных (15 баллов)
\item R3: Неавторизованный доступ (12 баллов)
\item R2: SQL инъекции (10 баллов)
\end{itemize}

\section*{Заключение}

Документация полностью соответствует критериям:
\begin{itemize}
\item 8 NFR с критериями ✓
\item 3-4 DFD с границами доверия ✓
\item 16 угроз STRIDE ✓
\item Полная трассируемость мер ✓
\item Четкая приоритизация рисков ✓
\end{itemize}

\end{document}
